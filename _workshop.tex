\documentclass[sigchi-a, authorversion]{acmart}
\usepackage{booktabs} % For formal tables
\usepackage{ccicons}  % For Creative Commons citation icons

% Copyright
%\setcopyright{none}
%\setcopyright{acmcopyright}
%\setcopyright{acmlicensed}
\setcopyright{rightsretained}
%\setcopyright{usgov}
%\setcopyright{usgovmixed}
%\setcopyright{cagov}
%\setcopyright{cagovmixed}


% DOI
\acmDOI{10.475/123_4}

% ISBN
\acmISBN{123-4567-24-567/08/06}

%Conference
\acmConference[CHI'19]{ACM CHI conference}{May 2019}{Glasgow, UK}
\acmYear{2019}
\copyrightyear{2019}

\acmPrice{15.00}

%\acmBadgeL[http://ctuning.org/ae/ppopp2016.html]{ae-logo}
%\acmBadgeR[http://ctuning.org/ae/ppopp2016.html]{ae-logo}

\begin{document}
\title{Transparent Research at CHI}

\author{Chat Wacharamanotham}
\affiliation{%
  \institution{University of Zurich }
  \city{Zurich}
  \country{Switzerland} 
  }
\email{chat@ifi.uzh.ch}

\author{Matthew Kay}
\affiliation{%
  \institution{School of Information, University of Michigan}
  }
\email{mjskay@umich.edu}

\author{Xiaoying Pu}
\affiliation{%
  \institution{Computer Science and Engineering, University of Michigan}
  }
\email{xpu@umich.edu}

\author{Ulrik Lyngs}
\affiliation{
  \institution{Department of Computer Science, University of Oxford}
  }
\email{ulrik.lyngs@cs.ox.ac.uk}

\author{Shion Guha}
\affiliation{
  \institution{Marquette University}
  \city{Milwaukee, WI}}
\email{shion.guha@marquette.edu}


% The default list of authors is too long for headers.
\renewcommand{\shortauthors}{F. Author et al.}


%
% The code below should be generated by the tool at
% http://dl.acm.org/ccs.cfm
% Please copy and paste the code instead of the example below.
%
\begin{CCSXML}
<ccs2012>
<concept>
<concept_id>10003120.10003121.10003122</concept_id>
<concept_desc>Human-centered computing~HCI design and evaluation methods</concept_desc>
<concept_significance>500</concept_significance>
</concept>
</ccs2012>
\end{CCSXML}

\ccsdesc[500]{Human-centered computing~HCI design and evaluation methods}


\begin{abstract}
Research transparency includes the dissemination of research materials, data, and decisions made throughout the course of research. While it has been recognized by other scientific disciplines, there are debates in HCI on the extent that transparency should be incorporated in the research and reviewing process. We propose a one-day writing workshop to develop guideline documents that collect consensus and controversies from the CHI community in how research transparency can be applied in HCI.
\end{abstract}


\keywords{Transparency; methodology; preregistration; reproducibility.}



\maketitle

%\begin{sidebar}
%  \textbf{Good Utilization of the Side Bar}
%
%  \textbf{Preparation:} Do not change the margin
%  dimensions and do not flow the margin text to the
%  next page.
%
%  \textbf{Materials:} The margin box must not intrude
%  or overflow into the header or the footer, or the gutter space
%  between the margin paragraph and the main left column.
%
%  \textbf{Images \& Figures:} Practically anything
%  can be put in the margin if it fits. Use the
%  \texttt{{\textbackslash}marginparwidth} constant to set the
%  width of the figure, table, minipage, or whatever you are trying
%  to fit in this skinny space.
%
%  \caption{This is the optional caption}
%  \label{bar:sidebar}
%\end{sidebar}

%\begin{figure}
%  \includegraphics[width=\marginparwidth]{sigchi-logo}
%  \caption{Insert a caption below each figure.}
%  \label{fig:sample}
%\end{figure}


\section{Background}
Research transparency has been recognized as a core value in many scientific disciplines \cite{Nosek2015}. HCI---being interdisciplinary---already has several sets of core values, many of which may need to be weighed against transparency, e.g., aesthetic accountability in design research, researchers' subjective interpretation in qualitative research, and the ability to cope with real-world complexity in invention research \cite[~ch. 7, 8, and 4, respectively]{Olson2014}). Furthermore, research transparency in HCI can be constrained by practical and business concerns, e.g., data privacy and intellectual property. Finally, the HCI review and publication processes have yet to support best practices in research transparency (e.g., preregistration and data sharing). The extent to which HCI research should embrace transparency is a subject of an ongoing discussion, including in various venues at CHI 2018 \cite{Chuang2018, Cockburn2018, Echtler2018, Wacharamanotham2018}


This workshop aims to collect the consensus, controversies, and best practices (from within and outside HCI) in research transparency. We plan a \emph{working workshop}: we will solicit participants to work for the majority of the workshop time on developing guideline documents for a specific set of topics. Topics and the assignment of participants to topics will be determined in advance based on participants' stated interests in their responses to our CfP, in order to maximise working time at the workshop. The workshop will culminate in a set of documents that can be used as a basis for the HCI community to navigate strategies to incorporate research transparency as a core value in HCI research. Where appropriate, these documents will be incorporated as chapters in the Transparent Statistics in HCI Guidelines  (\url{transparentstats.github.io/guidelines/}) \cite{TransparentStats2018}. We have identified five potential topics that will be worked on in the workshop. In the following, we give an overview of each area along with starting points for discussion.


\subsection{Preregistration and experiment planning}
As a way of being transparent about the research process, preregistration distinguishes exploratory analyses and results from confirmatory ones \cite{Nosek2018}. Preregistration was proposed as a tool to help address the replication crisis; other disciplines, such as psychology, are rapidly adopting this practice \cite{Kupferschmidt1192}. Another advantage of preregistration is that it encourages authors to be more explicit when planning their experimental designs. In HCI, Cockburn et al. discussed how preregistration can be adapted to experimental HCI and proposed an HCI-specific registry for preregistration \cite{Cockburn2018}. During the workshop, we would like to develop guidelines around preregistration and experiment planning in HCI, including but not limited to: (1) how to handle logistics for publishing and reviewing preregistered studies, such as anonymization and idea scooping; (2) how to review preregistered studies, including using preregistration to aid review and how to handle deviations from preregistered protocols; (3) how to weigh trade-offs between alternative experimental designs and how to document and publish such design decisions. We also aim to discuss (1) how the option of preregistration can benefit HCI researchers who do non-experimental work, or exploratory and iterative studies in general; (2) how to ensure that the promotion of preregistration does not unintentionally bias against other types of research contributions (e.g., \emph{will certain types of studies be favored or penalized if SIGCHI promotes preregistration guidelines? If so, how do we prevent this?}); and (3) how to lower the barrier to adoption for preregistration, for example by adapting existing preregistration templates (such as ones on \href{http://osf.io}{OSF.io} and \href{http://aspredicted.org}{AsPredicted.org}) to HCI studies.

%===============================================================================
\begin{sidebar}
\vspace{-5.5cm}
\section{Organizers (1/2)}
%===============================================================================

Three co-organizers of this workshop (Wacharamanotham, Kay, and Guha) have previously organized SIGs and a workshop at CHI 2016, '17, and '18 which focused on transparent statistics \cite{Kay2016a,Kay2017a,Wacharamanotham2018}. \\

\textbf{Chat Wacharamanotham} is an Assistant Professor at the University of Zurich. He studies how scientists use statistics, both in conducting statistical analysis and in consuming statistical reports. He develops and study interactive tools for conducting statistical analysis \cite{Wacharamanotham2015} and learn statistics. He can be found online at: \url{zpac.ch/chat}. \\

\textbf{Matthew Kay} is an Assistant Professor in the University of Michigan School of Information working in human-computer interaction and information visualization. He studies the communication of uncertainty in domains like personal informatics, everyday sensing and prediction, and scientific communication. He has published work advancing the use of Bayesian statistics in VIS and CHI. His website is: \url{www.mjskay.com}. \\

\textbf{Xiaoying Pu} is a Ph.D. student in Computer Science and Engineering at the University of Michigan. She designs and evaluates visualizations to make user interactions more statistically reliable. She also has a keen interest in adopting and improving open science practices. Her website is \url{xiaoyingpu.github.io}. \\

(continues on next page)
\end{sidebar}
%===============================================================================


\subsection{Research material and data transparency}
Making \emph{research materials} and \emph{data} publicly available allows other researchers to scrutinize and follow up. The HCI field lacks consensus on which materials should be made public at the time of publication, where they should be stored, and how to manage access to them. Moreover, much HCI research involves collecting data that are personal to study participants. Before making the data public, researchers need to address participants' privacy by anonymization and/or aggregation---both approaches potentially remove important context from the data. Data aggregation may even reduce transparency. On the other hand, at the point of data collection, informing participants (e.g., in an informed consent) that the data will be made available may deter potential participants (especially in sensitive populations), cause selection bias, or influence their responses. Researchers must balance these tradeoffs when designing studies and deciding how or whether to release their data.


Reviewers also face several conundrums: (1) To what extent should supplementary research materials be scrutinized during the review process? (2) How should we ensure fairness of the judgment between the submissions that make their materials and data available (for potential in-depth scrutiny) versus those that do not (per the submission tradition). The field of HCI as a whole also faces questions such as (1) How to incentivize researchers to publish their research materials? (2) How to encourage industrial research labs to share their data (despite intellectual property concerns)? In the workshop, we would like to discuss these issues from both the authors' and the reviewers' perspective. From these discussions we aim to produce an initial guideline to encourage the sharing of research materials and data while respecting other values such as data privacy and intellectual properties.





\subsection{Reproducible analytic workflows}

Reproducible workflows allow reviewers, other researchers and the authors themselves to retrace the steps taken during data analysis and reporting. 

There is little consensus on how authors should make their data analysis workflows accessible and transparent. As a result, reviewers and readers oftentimes cannot reproduce analyses reported in papers---despite the data being provided. Although tools for automating the transfer of results from statistical software to manuscript exist (e.g., RMarkdown), it is unclear how the analysis should be organized between the actual manuscript and further supplementary materials (e.g., \emph{where should descriptive statistics and tests of statistical assumptions be presented? How detailed should they be?}).

During the workshop, we will collaboratively work on guidelines for: How to (1) document workflows involved in data management and analysis, (2) document workflows involved in producing the results presented in publications, (3) document and coordinate analyses presented in the main publication text with those submitted in supplementary materials.

%===============================================================================
\begin{sidebar}
\vspace{-5.5cm}
\section{Organizers (2/2)}
%===============================================================================

\textbf{Ulrik Lyngs} is a PhD student at the Department of Computer Science, University of Oxford. His research explores how insights from the behavioural neurosciences may be applied to design digital devices that are sensitive to human cognitive limitations and biases, particularly in relation to attention and self-regulation.  Ulrik has blogged about how to write reproducible papers for CHI using R Markdown in combination with the ACM LaTeX template. His website is: \url{www.ulriklyngs.com}. \\
 
\textbf{Shion Guha} is an Assistant Professor of Computer Science and Director of Data Science at Marquette University. He studies the different aspects of privacy in social networks, often from the lens of developing and marginalized contexts. More recently, his work has focused on algorithmic accountability, transparency and harm, particularly in crime analysis. He has recently published methodological papers of interest to the HCI community in JASIST, GROUP, and CSCW. His website is: \url{www.shionguha.net}. \\

\section{Link to Website}
From the previous SIGs and workshop, we have a website (\url{transparentstatistics.org}),  mailing list / Google group (\url{tinyurl.com/transparent-stats-group}), and a Slack group (\url{transparentstatistics.slack.com}). For the workshop, we will update the website to include the workshop call for participation and instructions for how participants can prepare in advance of the workshop.

\end{sidebar}
%===============================================================================



\subsection{Transparency in qualitative research}

Qualitative research is an integral part of HCI research \cite{Cairns2008} and ways of knowing \cite{Olson2014}. To their detriment, some discussions of research transparency have not engaged with qualitative research.  Qualitative research is interpretive in nature \cite{Boehner2007} and thus, research transparency is  important to be able to understand the methods and context of the work. Similar to conversations in quantitative research, qualitative methodologists have started talking about what transparency in qualitative methods means for their particular disciplines \cite{Tong2012,Obrien2014, Moravcsik2014}. 

However, this discussion is underexplored in HCI research contexts. We would like to use this workshop to work towards developing guidelines for transparency in qualitative research in HCI from both authors' and reviewers' perspectives. Some of this discussion will intersect with other groups in the workshop, e.g. \emph{research material and data transparency} described above. Other discussions will be specific to qualitative research. For instance, (1) how do existing transparency norms in qualitative research affect current HCI qualitative research practices? \cite{Schwandt2007} or (2) what practical strategies exist for promoting transparency in HCI qualitative research practices? \cite{Shenton2004} All of this is further complicated when qualitative research is conducted in developing and marginalized contexts where hereto ``western'' standards of transparency may not be completely applicable \cite{Ahmed2017}. For instance, in prior work on studying mobile privacy and security among low literate populations in Bangladesh \cite{Ahmed2017}, the authors found that due to religious objections, women would often be accompanied by their husbands during interviews. Many times, husbands answer for their wives after the latter whisper their thoughts to their husband as they may feel shy if interviewed by a male interviewer. From the western perspective, this action goes against norms of \emph{credibility} and \emph{dependability} as described in current thinking about transparency in qualitative research \cite{Shenton2004}. However, when taking Bangladeshi cultural context into account, the research transparency is still intact. In general, while qualitative research in HCI has developed certain norms about reporting and analysis, these norms may not necessarily apply under all research contexts. We see this as a major, holistic opportunity to integrate and inform qualitative and quantitative research transparency in HCI.



%===============================================================================
\begin{sidebar}
\vspace{-1.8cm}

%===============================================================================

\section{Call for Participation (1/2)}
HCI is multidisciplinary, drawing on a variety of research methodologies, including design, engineering, ethnography, empirical observation, and experimentation. Many academic disciplines increasingly recognize \textbf{research transparency} as a key characteristic of high-quality research. However, in the field of HCI, we lack a consensus on how to practice and review transparency in research, the support and incentives for transparency in the reviewing and publication process, and the educational resources that are suitable for the nature of HCI research. \\

We are running \textbf{a working workshop to develop guidelines for practicing research transparency in HCI}. Participants will work in groups to brainstorm and write guidelines for helping authors to practice research transparency and reviewers to evaluate research transparency. We will also develop concrete suggestions for changes to review processes and publication infrastructure. These documents will be shared publicly once finalized and we will engage the CHI Executive Committee and/or Program Chairs to discuss the feasibility of implementing the suggested changes. \\

We will invite participants through our transparent statistics mailing list and Slack (80+members); the ACM SIGCHI, ACM SIGCHI Research Ethics, and CHI-Meta Facebook groups (> 2500 members); and various departmental mailing lists. Since research transparency had been widely \\

(continues on next page)
\end{sidebar}






%===============================================================================
\section{Pre-Workshop Plans}
%===============================================================================



Based on position papers, we will determine a small number of topical groups decided in advance of the workshop. The accepted workshop position papers will be circulated among the participants in advance to allow them to get to know each other's background.

Prior the workshop, we will invite participants to share ideas and resources through a dedicated channel on the Transparent Statistics Slack. External contributors who could not attend the workshop can also contribute via this channel.



%===============================================================================
\section{Workshop Structure}
%===============================================================================

To facilitate collaborative writing, for the majority of the workshop we will break participants into 3--5 writing groups of \textasciitilde 3--5 people each to work on specific artifacts in separate Google documents. The structure will be as follows:

\begin{itemize}
\item 20 minutes. Introduction to workshop, overview of goals and structure of workshop, grouping.
\item 2 hours. Based on participants' expertise and interests (which we will balance in advance using position statements), participants will break into small groups to work on specific topics. Groups will have an organizer designated to suggest deliverables and to record consensus, controversies, and ideas within each group that should be raised to the plenary discussion.
\item 45 minutes plenary discussion. The groups report topics that require broader discussion, especially concerns that may cross-cut the initial groups. Participants will be encouraged to self-organize the groups for the afternoon session.
\item Lunch
\item 2 hours. More group writing. If desired, participants may rotate groups.
\item 1 hour plenary discussion (as above). Again focus on topics needing broader discussion, with an additional focus on steps needed to finalize artifacts.
\item 30 minutes. Closing discussion of concrete next steps, soliciting volunteers for follow-up tasks.
\end{itemize}



%===============================================================================
\section{Post-Workshop Plans}
%===============================================================================

As stated above, we plan a working workshop: the first draft of guidelines under each topic in the Background will be our primary outcome. We plan to publish this in the format that is suitable for the content. Firstly, quantitative guidelines will be integrated into the Transparent Statistics Guideline (\url{transparentstats.github.io/guidelines/}) \cite{TransparentStats2018}---which covers topics related to transparent practice of statistics in HCI community. The transparent statistics community had iteratively refined the content of this guideline already for three years. It is open-source (\url{github.com/transparentstats/guidelines}). Each version of the guideline is permanently archived and citable with specific DOI (\url{zenodo.org/record/1228032}). We have established the process for drafting, reviewing, and endorsing content of the guideline (\url{github.com/transparentstats/guidelines/wiki}). We will use this process to elicit public comments and reach out to the CHI community by posting to CHI-related mailing lists, the transparent statistics mailing list, and the ACM SIGCHI, ACM SIGCHI Research Ethics, and CHI-Meta Facebook group. We will establish a similar site for qualitative guidelines.

Secondly, since a few of our outcomes may involve recommendations of changes or augmentations to the review process, we will also engage the CHI Executive Committee and/or Program Chairs for future iterations of the conference to discuss the feasibility of changes to the review process, and the possibility of incorporating some of our artifacts into (or referencing those artifacts from) official documents like the SIGCHI reviewer guidelines. To date, we already engaging with CHI 2020 Technical Program Chairs (Andy Cockburn and Joanna McGrenere) with a proposal to revise CHI guideline for the authors and reviewers in the aspect of Replicability and Transparency (\url{tinyurl.com/chi-guideline-replicability}), and we have proposed the formation of a special committee on transparent statistical communication with the SIGCHI Executive Committee (with positive response, due to begin this year).

Thirdly, for issues that have not reached consensus in the workshop, we will attempt to publish a report in a magazine that reaches a broader swath of the community, such as Interactions or Communications of the ACM. Such publication will provide a basis of further discussion about research transparency in HCI.

%===============================================================================
\begin{sidebar}
\vspace{-10cm}
%===============================================================================

\section{Call for Participation (2/2)}

discussed at CHI 2018 (in both our SIG and the Transparency and Openness Promotion SIG) and in other research communities, we are confident we can attract a motivated and knowledgeable set of participants. All of our previous SIGs and workshop have been well-attended, and the previous workshop spawned the transparent statistics guidelines we continue to expand upon (\url{https://transparentstats.github.io/guidelines/}).


We will focus on topics including preregistration and experiment planning, research material and data transparency, reproducible analytic workflows, and transparency in qualitative research. \\

\textbf{We are looking for a diverse set of perspectives on research methods} in the HCI community to develop these documents. If you are interested in research transparency in HCI, submit a position statement (at most 2 pages in CHI Extended Abstracts format) containing a short bio, a position statement, and an indication of which topics that you prefer to contribute, or a suggestion for another topic you believe should be included at the workshop.

\end{sidebar}





\bibliography{_references}
\bibliographystyle{ACM-Reference-Format}

\end{document}
